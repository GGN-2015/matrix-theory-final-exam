\documentclass{article}

\usepackage{amsmath}
\usepackage{amssymb}
\usepackage{ctex}
\usepackage{datetime2}
\usepackage{graphicx} % Required for inserting images
\usepackage{hyperref}
\usepackage{shellesc}
\usepackage{tabularx}
\usepackage{xcolor}

\newcommand{\getcompiletimestamp}{
	编译时间:\DTMnow
}

\title{2024 秋 (研) 矩阵理论-I\\期末考试试题整理}
\author{GGN\_2015}
\date{}

\begin{document}

\maketitle

\section{声明}

本答案由学生 GGN\_2015 自制\footnote{\getcompiletimestamp},正确性并无保证.如果出现了谬误,欢迎联系 premierbob@qq.com 更正.本文题目为 2024 春季期末考试试题与考点分析,欢迎同学们一同分享学习的心得体会.本项目的 github 地址为:\href{https://github.com/GGN-2015/matrix-theory-final-exam}{https://github.com/GGN-2015/matrix-theory-final-exam},欢迎 Pull Request. 另外,项目 \href{https://github.com/GGN-2015/2025-spring-matrix-theory}{https://github.com/GGN-2015/2025-spring-matrix-theory} 中公开了本文作者的全部课程笔记。



\newpage

\tableofcontents

\newpage

\section{判断题}

\subsection{T1}

\par 设 \(V=\{A\in \mathbb R^{3\times 3}| \det(A) = 0\}\),则 $V$ 在 $\mathbb R$ 上定义的普通矩阵加法和数乘下构成线性空间.

\par \textbf{答案}:错误

\par \textbf{理由}:两个奇异矩阵($\det(A)=0, \det(b)=0$)的加和可能非奇异($\det(A+B)\neq 0$),因此在 $V$ 上的普通矩阵加法,不满足运算的封闭性. 一个具体的例子:$A=\begin{pmatrix}
	1 & 0\\
	0 & 0
\end{pmatrix}, B = \begin{pmatrix}
	0 & 0\\
	0 & 1
\end{pmatrix}, A+B=I$.

\subsection{T2}

\par 在有限维线性空间中,同一个向量在不同的基下的坐标可能相同.

\par \textbf{答案}:正确

\par \textbf{理由}:例如,如果 $\{e_1, e_2\}$ 是 $\mathbb R^2$ 的一组基,那么 \(\{e_1, -e_2\}\) 也一定是 $\mathbb R^2$ 的一组基.那些与 $e_1$ 平行的向量,在两组基底下一定具有相同的坐标.

\subsection{T3}

\par 设 $n$ 阶复方阵 A、B 具有相同的特征多项式和最小多项式,则 A 和 B 一定相似.

\par \textbf{答案}:错误

\par \textbf{理由}:两个矩阵特征多项式相同,说明两个矩阵具有相同的特征值,且各个特征值的重数对应相等.两个矩阵具有相同的最小多项式,这说明两个矩阵各自的约当标准形中,每个特征值 $\lambda_i$ 的最大 Jordan 块的阶数相同.由于相似的充要条件是,两个矩阵具有相同的约当标准型,但以上的两个条件仅仅限制了约当标准型中每个特征值的总次数和最大块阶数,因此并不能说明两个矩阵相似.

\par \textbf{例子}:

\begin{equation*}
	\begin{pmatrix}
		\begin{matrix}
			\boxed{1} & 0\\
			0 & \boxed{1}\\
		\end{matrix} &
		\begin{matrix}
			0 & 0\\
			0 & 0\\
		\end{matrix}\\
		\begin{matrix}
			0 & 0\\
			0 & 0\\
		\end{matrix} &
		\boxed{\begin{matrix}
				1 & 1\\
				0 & 1\\
		\end{matrix}}\\
	\end{pmatrix} \nsim
	\begin{pmatrix}
		\boxed{\begin{matrix}
				1 & 1\\
				0 & 1\\
		\end{matrix}}&
		\begin{matrix}
			0 & 0\\
			0 & 0\\
		\end{matrix}\\
		\begin{matrix}
			0 & 0\\
			0 & 0\\
		\end{matrix}&
		\boxed{\begin{matrix}
			1 & 1\\
			0 & 1\\
		\end{matrix}}\\
	\end{pmatrix}
\end{equation*}

但他们的特征多项式都是 $(x-1)^4$ 最小多项式都是 $(x-1)^2$,矩形框圈出了上述两个矩阵的约当小块.

\subsection{T4}

\par 设 $A、B$ 均为 $n$ 阶段复方阵,则 $R(AB)\subseteq R(A), N(B)\subseteq N(AB)$.

\par \textbf{答案}:正确

\par \textbf{理由}:首先, $R(A)$ 表示线性映射 $A$ 的值域(range),$N(A)$ 表示线性映射 $A$ 的核(kernel).从概念上讲 $R(A)=\{Ax | x\in \mathbb C^n\}, N(A)=\{x\in \mathbb C^n|Ax=0\}$.不难验证上面的包含关系.

\subsection{T5}

\par 若 $A$ 是 $n$ 阶可逆复方阵,则 $\sin A$ 一定可逆.

\par \textbf{答案}:错误

\par \textbf{理由}:$n$ 阶段复方阵可逆当且仅当不存在零特征值.设 $\lambda(A)=\{\lambda_1, \cdots, \lambda_n\}$,则根据特征值的遗传公式 $\lambda(\sin A)=\{\sin \lambda_1, \cdots, \sin \lambda_n\}$.即使 $\lambda_i$ 中没有零,但当某个 $\lambda_i=k\pi\;(k\neq 0)$ 时,$\sin A$ 也不可逆.

\subsection{T6}

\par 设 $T$ 是 $n$ 维欧氏空间 $V$ 的一个正交变换,则 $T$ 对应的矩阵一定是正交矩阵.

\par \textbf{答案}:正确

\par \textbf{理由}:正交矩阵的定义是 $AA^T=A^TA=I_n$.正交变换的定义是 $\forall x, y\in V, (A(x), A(y))=(x, y)$ 即保内积,即 $(Ay)^T(Ax)=y^T(A^TA)x=y^Tx$,总成立.由于 $x, y$ 的任意性,所以得到 $A^TA=I_n$.

\subsection{T7}

\par 设 $A、B$ 均是正规复矩阵,且 $AB=BA$,则 $AB$ 和 $BA$ 也均是正规矩阵.

\par \textbf{答案}:正确

\par \textbf{理由}:由于 $A$ 与 $B$ 正规且可换,根据同时对角化定理,则存在一个酉矩阵 $Q$ 使得 $Q^{-1}AQ$ 与 $Q^{-1}BA$ 都是对角阵.这说明 $Q^{-1}ABQ = Q^{-1}AQ\times Q^{-1}BQ$ 也是对角阵.因此 $AB$ 能够酉相似对角化,因此 $AB$ 是正规矩阵.

\subsection{T8}

\par 设 $A$ 是 $n$ 阶复方阵,则 $\|A\|_2\leq \|A\|_F$.

\par \textbf{答案}:正确

\par \textbf{理由}:$\|A\|_2$ 是 矩阵 $A$ 的谱半径,即 $\sqrt{\sigma_{\text{max}}(A^HA)}$,可以证明,对于 $n$ 阶复方阵,谱半径是所有范数的下界.

\subsection{T9}

\par 设 $A=\begin{pmatrix}-1 & 1 & 0\\0 & -1 & 0\\0 & 0 &-1\end{pmatrix}$,则级数 $\sum_{k=1}^{\infty}\frac{1}{k^2}A^k$ 不绝对收敛.

\par \textbf{答案}:正确

\par \textbf{理由}:矩阵级数绝对收敛的条件是,谱半径小于级数收敛半径.$A$ 的谱半径为 1,级数的收敛半径也为 1,因此不绝对收敛.

\subsection{T10}

\par $\forall A\in \mathbb C^{n\times n}, x\in \mathbb C^n$,若向量范数 $\|x\|_v$ 与矩阵范数 $\|A\|_m$ 满足不等式 $\|Ax\|_v\geq \|A\|_m \|x\|_v$ 则称矩阵范数 $\|A\|_m$ 与向量范数 $\|x\|_v$ 相容.

\par \textbf{答案}:错误

\par \textbf{理由}:不等号方向不对,应当是 $\|Ax\|_v\leq \|A\|_m \|x\|_v$.

\section{填空题}

\subsection{T1}

\par 设 $A=\begin{pmatrix}
	1 & 2 & 1 & 0\\
	1 & 1 & 1 & 1
\end{pmatrix}, B = \begin{pmatrix}
	2 & -1 & 0 & 1\\
	1 & -1 & 3 & 7
\end{pmatrix}$,$V_1, V_2$ 分别是齐次方程组 $Ax=0$ 和 $Bx=0$ 的解空间,则 $\dim(V_1\cap V_2)=\underline{\phantom{empty\_space}}$.

\par \textbf{答案}:1

\par \textbf{理由}:本质上就是求以下方程的解空间的维数:

\begin{equation*}
	\begin{pmatrix}
		1 & 2 & 1 & 0\\
		1 & 1 & 1 & 1\\
		2 & -1 & 0 & 1\\
		1 & -1 & 3 & 7
	\end{pmatrix} \begin{pmatrix}
		x_1\\x_2\\x_3\\x_4
	\end{pmatrix}=0
\end{equation*}

\par 通过高斯消元,可以证明矩阵 $\begin{pmatrix}
	1 & 2 & 1 & 0\\
	1 & 1 & 1 & 1\\
	2 & -1 & 0 & 1\\
	1 & -1 & 3 & 7
\end{pmatrix}$ 的秩为 3,因此 $\dim(V_1\cap V_2)=1$.

\subsection{T2}

\par 已知向量组 $\alpha_1, \alpha_2, \alpha_3$ 线性无关,若向量组 $\alpha_1+\alpha_2, \alpha_2+\alpha_3, \alpha_3+k\alpha_1$ 也线性无关,则 $k$ 应当满足 \underline{\phantom{empty\_space}}.

\par \textbf{答案}:$k\neq -1$

\par \textbf{理由}:只需要保证 $a(\alpha_1+\alpha_2) + b(\alpha_2+\alpha_3)+c(\alpha_1+k\alpha_3)=0$ 当且仅当 $a=b=c=0$.整理得到 $(a+ck)\alpha_1 + (a+b)\alpha_2 + (b+c)\alpha_3=0$ 当且仅当 $a=b=c=0$.由于 $\alpha_1, \alpha_2, \alpha_3$ 线性无关,因此得到 $(a+ck)\alpha_1 + (a+b)\alpha_2 + (b+c)\alpha_3=0$ 等价于 $a+ck=0$ 且 $a+b=0$ 且 $b+c=0$,因此我们只需要保证以下方程有唯一解:

\begin{align*}
	\begin{pmatrix}
		1 & 0 & k\\
		1 & 1 & 0\\
		0 & 1 & 1
	\end{pmatrix} \begin{pmatrix}
		a\\b\\c
	\end{pmatrix}=0
\end{align*}

保证上述矩阵满秩即可.

\subsection{T3}

\par 设 $M=I_n - ww^T, w=\frac{1}{\sqrt n}[ 1, 1, \cdots, 1]^T \in \mathbb R^n$, 则 $M$ 的行列式等于 \underline{\phantom{empty\_space}}.

\par \textbf{答案}:0

\par \textbf{理由}:换位公式对任意 $A_{n\times p}, B_{p\times n}$ 总有关于 $\lambda$ 的两个多项式 $\det(\lambda I_n - AB) \cdot \lambda^p \equiv \det(\lambda I_p - BA) \cdot \lambda^n $.带入 $\lambda = 1$ 得到 $\det(I_n - AB) = \det(I_p - BA)$, 令 $A=w, B=w^T, p=1$, 得到 $\det(I_n - ww^T)=1 - w^Tw=0$.

\subsection{T4}

\par 设 $A=\begin{pmatrix}
	2 & 1 & 1\\
	1 & 2 & 1\\
	1 & 1 & 2
\end{pmatrix}, x=\begin{pmatrix}
	1\\0\\1
\end{pmatrix}$,则 $\|A\|_1=\underline{\phantom{empty\_space}}, \|Ax\|_{\infty}=\underline{\phantom{empty\_space}}, \|A\|_F=\underline{\phantom{empty\_space}}$.

\par \textbf{答案}:$4, 3, 3\sqrt 2$

\par \textbf{理由}:一些常见的范数定义如下:

\begin{tabularx}{\linewidth}{|X|X|X|X|X|} 
	\hline
	 & 1-范数 & 2-范数 & F-范数 & $\infty$-范数 \\ 
	\hline
	矩阵范数 & 最大列和范数 & 谱半径 & 二次方范数 & 行和范数 \\ 
	\hline
	向量范数 & 所有元素绝对值之和 & 二次方范数 & 不存在 & 元素最大绝对值 \\ 
	\hline
\end{tabularx}

\par 这个对应关系看起来有些奇怪,但实际上表达出一种“诱导关系”.即向量 1-范数 诱导的算子范数是矩阵 1-范数、向量 2-范数诱导的算子范数是矩阵 2-范数,依此类推.

\subsection{T5}

\par 设 $A\in \mathbb R^{3\times 3}$,特征值 $\lambda=2$ 是三重特征根且只有一个线性无关的特征向量,则 $A$ 的 Jordan 标准型是 \underline{\phantom{empty\_space}}.

\par\textbf{答案}:$\begin{pmatrix}
	2 & 1 & 0\\
	0 & 2 & 1\\
	0 & 0 & 2
\end{pmatrix}$

\par \textbf{理由}:每个约当小块对应着一个线性无关的一维特征子空间,由于该矩阵只有一个线性无关的特征向量,因此不可能有大于等于两个约当小块.

\subsection{T6}

\par 设 $A=\begin{pmatrix}
	7 & 0 & 1\\
	-1 & -5 & i\\
	0 & 1 & 6i
\end{pmatrix}$,则 $A$ \underline{\phantom{empty\_space}}(填 “是” 或者 “不是”)单纯矩阵.

\par \textbf{答案}:是

\par \textbf{理由}:单纯阵的定义为可以相似对角化的矩阵,其充要条件是所有约当小块的阶数都为 1.因此,单纯阵的充分条件是,有 $n$ 个互不相同的特征值(因为这些特征值 ,每一个都至少有一个约当小块与之对应).观察到矩阵的对角线元素比较大,而其他元素比较小,使用盖尔圆盘对矩阵特征值进行估计.

第一行:$|\lambda_1-7|\leq |0| + |1|$,第二行:$|\lambda_2-(-5)|\leq |-1| + |i|$,第三行:$|\lambda_3-6i|\leq |0| + |1|$.这意味着三个盖尔圆盘没有交,因而矩阵有三个互不相同的特征值.

需要注意的是,盖尔圆盘估计对于行和估计与列和估计都是正确的,对于这个问题,使用列和估计则会出现圆盘有交的情况,不利于排除重根.

\subsection{T7}

\par $A=\begin{pmatrix}
	1 & -1 & 1 & 1\\
	-1 & 1 & -1 & -1\\
	-1 & -1 & 1 & 1\\
	1 & 1 & -1 & -1
\end{pmatrix}$,则 $A$ 的满秩分解为 \underline{\phantom{empty\_space}}.

\par \textbf{一种可行做法}:$\begin{pmatrix}
	1 & 0\\
	-1 & 0\\
	0 & 1\\
	0 & -1
\end{pmatrix} \begin{pmatrix}
	1 & -1 & 1 & 1\\
	-1 & -1 & 1 & 1
\end{pmatrix}$

\par \textbf{理由}:先计算得到原矩阵的秩为 2,从 $A$ 中任取两个线性无关的行作为 $C$, 再反推出 $B$,即可得到一个分解 $A=BC$.也可以先从 $A$ 从任选两个线性无关的列作为 $B$,再反推出 $C$.

\subsection{T8}

\par 设 $A=\begin{pmatrix}
	1 & 1\\
	0 & a
\end{pmatrix}, a\in \mathbb R$,若极限 $\lim_{k\to \infty}A^k$ 存在,则 $a$ 的取值范围是 \underline{\phantom{empty\_space}}.

\par \textbf{答案}:$(-1, 1)$

\par \textbf{理由}:首先若 $a=1$,$A^k$ 显然不收敛.当 $a\neq 1$ 时,矩阵 $A$ 有两个互异的特征值 $\lambda(A)={1, a}$,因此 $A$ 为单纯阵,可以相似对角化为 $P\times \text{diag}(1, a)\times P^{-1}$.于是 $A^k=P\times \text{diag}(1, a^k)\times P^{-1}$,于是当 $a\in(-1, 1)$ 时 $A^k$ 收敛.

\subsection{T9}

\par 设 $A=\begin{pmatrix}
	1 & 1 & 1\\
	0 & 1 & 1\\
	0 & 0 & 1
\end{pmatrix}$,定义集合 $S_A=\{\sum_{i=1}^\infty a_iA^i|\forall a_i\in \mathbb R, \text{finite\;many\;} a_i\neq 0\}$,则 $S_A$ \underline{\phantom{empty\_space}}(填 “是” 或者 “不是”)线性空间,如不是请给出理由 \underline{\phantom{empty\_space}} .

\par \textbf{答案}:是

\par \textbf{理由}:不难观察到 $\lambda(A)=\{1, 1, 1\}$,根据凯莱定理(特征多项式一定是矩阵的零化式),则一定有 $(A-I)^3=0$.整理得到:$A^3=3A^2-3A+I$,这说明对任意 $k>3$,$A^k$ 都能写成 $(I, A, A^2)$ 的线性组合,于是有 $S_A$ 次数不超过三次且封闭.

\section{大题}

\subsection{T3 (13分)}

\par 设 $\mathbb R^3$ 中的一组基为 $\alpha_1=(1, 0, -1)^T, \alpha_2=(2, 1, 1)^T, \alpha_3=(1, 1, 1)^T$,对 $\mathbb R^3$ 中任意的向量 $x=k_1\alpha_1 + k_2\alpha_2+k_3\alpha_3, k_j\in\mathbb R, j=1, 2, 3$,定义变换 $T$ 为 $T(x)=k_1T(\alpha_1)+k_2T(\alpha_2)+k_3T(\alpha_3)$,其中 $T(\alpha_1)=(0, 1, 1)^T, T(\alpha_2)=(-1, 1, 0)^T, T(\alpha_3)=(1, 2, 1)^T$ 试求解以下问题:

\begin{enumerate}
	\item 证明 $T$ 是线性变换;
	\item 求 $T$ 在 $\alpha_1, \alpha_2, \alpha_3$ 下的矩阵;
\end{enumerate}

\par \textbf{1.证明}:

只需证明 $T(ax + by)=aT(x)+bT(y)$,不妨设 $y=k'_1\alpha_1+k'_2\alpha_2+k'_3\alpha_3$;

左侧 $=T(ax+by)=T(\sum_{i=1}^3(ak_i + bk'_i)\alpha_i)=\sum_{i=1}^3(ak_i + bk'_i)T(\alpha_i)=a\sum_{i=1}^3k_iT(\alpha_i)+b\sum_{i=1}^3 k'_iT(\alpha_i)=$ 右侧

\par \textbf{2.解}

换基公式:$T(\alpha_1, \alpha_2, \alpha_3)=(\alpha_1, \alpha_2, \alpha_3) M(T)$,其中 $M(T)$ 是 $T$ 在基 $\alpha_1, \alpha_2, \alpha_3$ 下的矩阵.

首先,$T(\alpha_1)=-\alpha_2+2\alpha_3$,$T(\alpha_2)=\alpha_1-3\alpha_2+4\alpha_3$, $T(\alpha_3)=\alpha_1-2\alpha_2+4\alpha_3$

于是 $M(T)=\begin{pmatrix}
	0 & 1 & 1\\
	-1 & -3 & -2\\
	2 & 4 & 4
\end{pmatrix}$

\subsection{T4 (13分)}

\par 设 $A=\begin{pmatrix}
	1 & 4 & 2\\
	0 & -3 & 4\\
	0 & 4 & 3
\end{pmatrix}$

\begin{enumerate}
	\item 求 $A$ 的谱分解;
	\item 计算 $A^{100}$ (用谱阵表示即可);
\end{enumerate}

\par \textbf{1.解}

$A-I=\begin{pmatrix}
	0 & 4 & 2\\
	0 & -4 & 4\\
	0 & 4 & 2
\end{pmatrix}$ 这说明 $1$ 是 $A$ 的特征值,高低分解得到

$(A-I)=\begin{pmatrix}
	1 & 0\\
	0 & 1\\
	1 & 0
\end{pmatrix} \times \begin{pmatrix}
	0 & 4 & 2\\
	0 & -4 & 4
\end{pmatrix}$

我们交换高地阵,得到一个与 $(A-I)$ 具有相同非零特征值的矩阵

$ \begin{pmatrix}
	0 & 4 & 2\\
	0 & -4 & 4
\end{pmatrix}\times \begin{pmatrix}
1 & 0\\
0 & 1\\
1 & 0
\end{pmatrix}=\begin{pmatrix}
	2 & 4 \\
	4 & -4
\end{pmatrix}$ 解得该矩阵的特征值为 $\{-6, 4\}$

因此 $\lambda(A-I)=\{-6, 4 0\}$,因此 $\lambda(A)=\{-5, 5, 1\}$.不妨设 $A=(-5)\cdot G_1 + 5\cdot G_2 + 1\cdot G_3$ 是 $A$ 的谱分解:

解得:$G_1=\begin{pmatrix}
	0 & -\frac{2}{5} & \frac{1}{5}\\
	0 & \frac{4}{5} & -\frac{2}{5}\\
	0 & -\frac{2}{5} & \frac{1}{5}
\end{pmatrix}, G_2=\begin{pmatrix}
	0 & \frac{2}{5} & \frac{4}{5}\\
	0 & \frac{1}{5} & \frac{2}{5}\\
	0 & \frac{2}{5} & \frac{4}{5}
\end{pmatrix}, G_3=\begin{pmatrix}
	1 & 0 & -1\\
	0 & 0 & 0\\
	0 & 0 & 0
\end{pmatrix}$

\par \textbf{2.解}

根据 $G_1, G_2, G_3$ 的摄影性与正交性,可得: $A^{100}=(-5)^{100}G_1+5^{100}G_2 + G_3$

\subsection{T5 (13分)}

\par 已知矩阵 $A=\begin{pmatrix}
	-1 & 0 & 1\\
	1 & 2 & 0\\
	-4 & 0 & 3
\end{pmatrix}$,试求解以下问题:

\begin{enumerate}
	\item 求 $A$ 的 Jordan 标准型;
	\item 求矩阵的指数函数 $e^{tA}$;
\end{enumerate}

\par \textbf{1.解}

$A-I=\begin{pmatrix}
	-2 & 0 & 1\\
	1 & 1 & 0\\
	-4 & 0 & 2
\end{pmatrix}$,观察到第一行和第三行有倍数关系,因而矩阵降秩,于是 $1$ 是 $A$ 的特征值.为了进一步求解特征值,对 $A-I$ 进行高低分解:

$A-I=\begin{pmatrix}
	1 & 0\\
	0 & 1\\
	2 & 0
\end{pmatrix}\times \begin{pmatrix}
	-2 & 0 & 1\\
	1 & 1 & 0
\end{pmatrix}$ 由于我们的目的是计算非零特征值,所以我们交换高低阵:

$\begin{pmatrix}
	-2 & 0 & 1\\
	1 & 1 & 0
\end{pmatrix} \times \begin{pmatrix}
1 & 0\\
0 & 1\\
2 & 0
\end{pmatrix}=\begin{pmatrix}
	0 & 0\\
	1 & 1
\end{pmatrix}$ 于是说明 $\lambda(A-I)=\{0, 0, 1\}$,于是 $\lambda(A)=\{1, 1, 2\}$.这里我们发现有两个特征值 $1$,我们需要验证 $1$ 对应的特征向量空间维数,发现只有一维,这说明 $A$ 有一个二阶段约当块.

$J(A)=\begin{pmatrix}
	1 & 1 & 0\\
	0 & 1 & 0\\
	0 & 0 & 2
\end{pmatrix}$ 是 $A$ 的约当标准型.

\par \textbf{2.解}:对 $A$ 进行广义谱分解:

由于 $A$ 有半单-幂零分解,因此对于任意的多项式 $f(x)$ 一定存在 $f(A)=f(1)\cdot G_1 + f(2)\cdot G_2 + f'(1) \cdot D_1$.

带入 $f(x)=(x-2)$ 得到 $A-2I=- G_1+D_1$; 带入 $f(x)=(x-1)$ 得到 $A-I=G_2+D_1$;带入 $f(x)=(x-1)^2$ 得到 $(A-I)^2=G_2$;

解得 $G_1=\begin{pmatrix}
	1 & 0 & 0\\
	1 & 0 & -1\\
	0 & 0 & 1
\end{pmatrix}, G_2=\begin{pmatrix}
	0 & 0 & 0\\
	-1 & 1 & 1\\
	0 & 0 & 0
\end{pmatrix}, D_1\begin{pmatrix}
	-2 & 0 & 1\\
	2 & 0 & -1\\
	-4 & 0 & 2
\end{pmatrix}$

于是 $e^{tA}=e^tG_1+e^{2t}G_2+te^{t}D_1$.

\end{document}
