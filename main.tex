\documentclass{article}
\usepackage{ctex}
\usepackage{graphicx} % Required for inserting images
\usepackage{amssymb}
\usepackage{amsmath}

\title{北航研究生 2025 春季矩阵论期末复习}
\author{GGN\_2015}
\date{}

\begin{document}

\maketitle

\section{声明}

本答案由学生 GGN\_2015 自制,正确性并无保证。如果出现了谬误,欢迎联系 premierbob@qq.com 更正。本文题目为 2024 春季期末考试试题与考点分析,欢迎同学们一同分享学习的心得体会。本项目的 github 地址为:https://github.com/GGN-2015/matrix-theory-final-exam

\section{判断题}

\subsection{T1}

\par 设 \(V=\{A\in \mathbb R^{3\times 3}| \det(A) = 0\}\),则 $V$ 在 $\mathbb R$ 上定义的普通矩阵加法和数乘下构成线性空间。

\par \textbf{答案}:错误

\par \textbf{理由}:两个奇异矩阵($\det(A)=0, \det(b)=0$)的加和可能非奇异($\det(A+B)\neq 0$),因此在 $V$ 上的普通矩阵加法,不满足运算的封闭性。

\subsection{T2}

\par 在有限维线性空间中,同一个向量在不同的基下的坐标可能相同。

\par \textbf{答案}:正确

\par \textbf{理由}:例如,如果 $\{e_1, e_2\}$ 是 $\mathbb R^2$ 的一组基,那么 \(\{e_1, -e_2\}\) 也一定是 $\mathbb R^2$ 的一组基。那些与 $e_1$ 平行的向量,在两组基底下一定具有相同的坐标。

\subsection{T3}

\par 设 $n$ 阶复方阵 A、B 具有相同的特征多项式和最小多项式,则 A 和 B 一定相似。

\par \textbf{答案}:错误

\par \textbf{理由}:两个矩阵特征多项式相同,说明两个矩阵具有相同的特征值,且各个特征值的重数对应相等。两个矩阵具有相同的最小多项式,这说明两个矩阵各自的约当标准形中,每个特征值 $\lambda_i$ 的最大 Jordan 块的阶数相同。由于相似的充要条件是,两个矩阵具有相同的约当标准型,但以上的两个条件仅仅限制了约当标准型中每个特征值的总次数和最大块阶数,因此并不能说明两个矩阵相似。

\par \textbf{例子}:

\begin{equation*}
	\begin{pmatrix}
		\begin{matrix}
			\boxed{1} & 0\\
			0 & \boxed{1}\\
		\end{matrix} &
		\begin{matrix}
			0 & 0\\
			0 & 0\\
		\end{matrix}\\
		\begin{matrix}
			0 & 0\\
			0 & 0\\
		\end{matrix} &
		\boxed{\begin{matrix}
				1 & 1\\
				0 & 1\\
		\end{matrix}}\\
	\end{pmatrix} \nsim
	\begin{pmatrix}
		\boxed{\begin{matrix}
				1 & 1\\
				0 & 1\\
		\end{matrix}}&
		\begin{matrix}
			0 & 0\\
			0 & 0\\
		\end{matrix}\\
		\begin{matrix}
			0 & 0\\
			0 & 0\\
		\end{matrix}&
		\boxed{\begin{matrix}
			1 & 1\\
			0 & 1\\
		\end{matrix}}\\
	\end{pmatrix}
\end{equation*}

但他们的特征多项式都是 $(x-1)^4$ 最小多项式都是 $(x-1)^2$,矩形框圈出了上述两个矩阵的约当小块。

\subsection{T4}

\par 设 $A、B$ 均为 $n$ 阶段复方阵,则 $R(AB)\subseteq R(A), N(B)\subseteq N(AB)$。

\par \textbf{答案}:正确

\par \textbf{理由}:首先, $R(A)$ 表示线性映射 $A$ 的值域(range),$N(A)$ 表示线性映射 $A$ 的核(kernel)。从概念上讲 $R(A)=\{Ax | x\in \mathbb C^n\}, N(A)=\{x\in \mathbb C^n|Ax=0\}$。不难验证上面的包含关系。

\section{T5}

\par 若 $A$ 是 $n$ 阶可逆复方阵,则 $\sin A$ 一定可逆。

\par \textbf{答案}:错误

\par \textbf{理由}:方阵可逆当且仅当不存在零特征值。设 $\lambda(A)=\{\lambda_1, \cdots, \lambda_n\}$,则根据特征值的遗传公式 $\lambda(\sin A)=\{\sin \lambda_1, \cdots, \sin \lambda_n\}$。即使 $\lambda_i$ 中没有零,但当某个 $\lambda_i=k\pi\;(k\neq 0)$ 时,$\sin A$ 也不可逆。

\end{document}
